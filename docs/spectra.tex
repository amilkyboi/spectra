\documentclass[11pt, twoside, fleqn]{report}
\usepackage[backend=biber, style=ieee]{biblatex}
\usepackage[parfill, skip=1em]{parskip}
\usepackage[labelfont=bf]{caption}
\usepackage[margin=1in]{geometry}
\usepackage[hidelinks]{hyperref}
\usepackage{amssymb, booktabs, derivative, fancyhdr, float, graphicx, natex, pagecolor, pgfplots, siunitx, tikz}

% background and foreground colors
\definecolor{bgcolor}{HTML}{1e1e2e}
\definecolor{fgcolor}{HTML}{cdd6f4}

% catppuccin palette
\definecolor{p1}{HTML}{cba6f7}
\definecolor{p2}{HTML}{f38ba8}
\definecolor{p3}{HTML}{fab387}
\definecolor{p4}{HTML}{a6e3a1}
\definecolor{p5}{HTML}{89dceb}
\definecolor{p6}{HTML}{89b4fa}

% pagecolor
\pagecolor{bgcolor}
\color{fgcolor}

% biblatex
\addbibresource{references.bib}

% fancyhdr
\renewcommand{\chaptermark}[1]{\markboth{\thechapter\ #1}{}}
\renewcommand{\sectionmark}[1]{\markright{\thesection\ #1}}
\renewcommand{\footrulewidth}{0.4pt}

\fancypagestyle{mystyle}{
    \fancyfoot{}
    \fancyfoot[C]{\thepage}
}

% hyperref
\urlstyle{same}

% pgfplots
\pgfplotsset{compat=newest}

% custom
\newcommand{\dash}{\!-\!}
\newcommand{\state}[2]{\prescript{#1}{}{#2}}

\begin{document}

\pagestyle{empty}

\begin{titlepage}
    \null
    \vspace{\fill}
    \begin{center}
    \let \footnote \thanks
      {\LARGE \textbf{Simulation of Molecular Spectra} \par}
      {\large Theory and Applications}
      \vskip 1.5em
      {\large
        \lineskip .5em
        \begin{tabular}[t]{c}
          Nathan Phillips
        \end{tabular}\par}
      \vskip 1em
      {\large \today}
      \vskip 2em
    \end{center}
    \vspace{\fill}

    \begin{center}
        {\large{\textbf{Texas A\&M University}} \par}
        {\large{Department of Aerospace Engineering}}
    \end{center}
\end{titlepage}

\tableofcontents
\newpage
\listoffigures
\newpage
\listoftables
\newpage

\pagestyle{mystyle}

\input{chapters/approximations}

\input{chapters/structure}

\input{chapters/intensities}

\input{chapters/numbers}

\input{chapters/hund}

\input{chapters/uncoupling}

\input{chapters/transitions}

\input{chapters/lif}

\input{chapters/lineshapes}

\chapter{Placeholder}

\begin{align*}
    P_{11}(J) &= \nu_0^{(1)} + F_1'(J - 1) - F_1''(J) \\
    Q_{11}(J) &= \nu_0^{(1)} + F_1'(J) - F_1''(J)     \\
    R_{11}(J) &= \nu_0^{(1)} + F_1'(J + 1) - F_1''(J)
\end{align*}

\begin{align*}
    P_{12}(J) &= \nu_0^{(1)} + F_1'(J - 1) - F_2''(J) \\
    Q_{12}(J) &= \nu_0^{(1)} + F_1'(J) - F_2''(J)     \\
    R_{12}(J) &= \nu_0^{(1)} + F_1'(J + 1) - F_2''(J)
\end{align*}

\begin{align*}
    P_{22}(J) &= \nu_0^{(2)} + F_2'(J - 1) - F_2''(J) \\
    Q_{22}(J) &= \nu_0^{(2)} + F_2'(J) - F_2''(J)     \\
    R_{22}(J) &= \nu_0^{(2)} + F_2'(J + 1) - F_2''(J)
\end{align*}

\begin{align*}
    P_{21}(J) &= \nu_0^{(2)} + F_2'(J - 1) - F_1''(J) \\
    Q_{21}(J) &= \nu_0^{(2)} + F_2'(J) - F_1''(J)     \\
    R_{21}(J) &= \nu_0^{(2)} + F_2'(J + 1) - F_1''(J)
\end{align*}

\chapter{Multiplet Term Formulas}

Following info is from \textit{Rotational Structure in the Spectra of Diatomic Molecules} by Kov\'acs.

\section{General Multiplet Term Formulas}

In the following formulae, $B = \bar{B}/hc$, $D = \bar{D}/hc$ (p. 54).

These formulae are valid for any value of $\Lambda$ and $\Sigma$. However, they rarely give values that are compatible with experimental data (p. 57) since they are general in $S$. More useful formulae can be defined that are valid for any $Y$ and $\Lambda$, but fixed in $S$ (meaning different formulae for singlet, doublet, triplet, etc. splittings).

\subsection{Hund's Case (a)}

Equation 2.1.1.9
\begin{align*}
    F_a(\Lambda, S, Y \gg J(J + 1)) &= \nu_0 + A\Lambda\Sigma + B[J(J + 1) - \Omega^2 + S(S + 1) -\Sigma^2] \\
    &+ H_a^c(\Lambda, S) + H_a^{ss}(\Lambda, S) + H_a^{sr}(\Lambda, S)
\end{align*}

\subsection{Hund's Case (b)}

Equation 2.1.1.10
\begin{align*}
    F_b(\Lambda, S, Y \ll N(N + 1)) &= \nu_0 + B[N(N + 1) - \Lambda^2] + A\Lambda^2\frac{J(J + 1) - N(N + 1) - S(S + 1)}{2N(N + 1)} \\
    &+ H_b^c(\Lambda) + H_b^{ss}(\Lambda, S) + H_b^{sr}(\Lambda, S)
\end{align*}

\section{Singlet Terms}

For singlet terms, Hund's cases (a) and (b) are the same, and can therefore be obtained from Eq. 2.2.1.9 (by setting $S = \Sigma = 0$, or from 2.1.1.10 (by setting $S = 0$ and $J = N$).

Equation 2.1.2.1
\begin{equation*}
    F(J) = \nu_0 + B[J(J + 1) - \Lambda^2] - D[J(J + 1) - \Lambda^2]^2
\end{equation*}

\section{Triplet Terms}

For $\state{2}{\Sigma}$ terms, $\Lambda = 0$ and $Y = 0$ (p. 63).

Equation 2.1.3.15
\begin{align*}
    F_{J - \tfrac{1}{2}}(N) = F_1(N) &= \nu_0 + B_\Sigma N(N + 1) - D_\Sigma N^2(N + 1)^2 + \tfrac{1}{2}\gamma N \\
    F_{J + \tfrac{1}{2}}(N) = F_2(N) &= \nu_0 + B_\Sigma N(N + 1) - D_\Sigma N^2(N + 1)^2 - \tfrac{1}{2}\gamma(N + 1)
\end{align*}

\chapter{Intensity Distribution in Rotational Bands}

Following info is from \textit{Rotational Structure in the Spectra of Diatomic Molecules} by Kov\'acs.

\section{Triplet Transitions}

The following equations used in the intensity factors for triplets. For a $\state{3}{\Sigma}\dash\state{3}{\Sigma}$ transition, $\Lambda = 0$. For Hund's case (b), $Y$ can be replaced with $0$ (p. 70).

Equation 2.1.4.9
\begin{align*}
    u_1^\pm(J) &= \sqrt{\Lambda^2Y(Y - 4) + 4J^2} \pm \Lambda(Y - 2) \\
    u_3^\pm(J) &= \sqrt{\Lambda^2Y(Y - 4) + 4(J + 1)^2} \pm \Lambda(Y - 2)
\end{align*}

Equation 2.1.4.10
\begin{align*}
    C_1(J) &= \Lambda^2Y(Y - 4)(J - \Lambda + 1)(J + \Lambda) + 2(2J + 1)(J - \Lambda)J(J + \Lambda) \\
    C_2(J) &= \Lambda^2Y(Y - 4) + 4J(J + 1) \\
    C_3(J) &= \Lambda^2Y(Y - 4)(J - \Lambda)(J + \Lambda + 1) + 2(2J + 1)(J - \Lambda + 1)(J + 1)(J + \Lambda + 1)
\end{align*}

For a Hund's case (b) $\state{3}{\Sigma}\dash\state{3}{\Sigma}$ transition, these become
\begin{align*}
    u_1^\pm(J) &= 2J \\
    u_3^\pm(J) &= 2(J + 1)
\end{align*}

And
\begin{align*}
    C_1(J) &= 2J^3(2J + 1) \\
    C_2(J) &= 4J(J + 1) \\
    C_3(J) &= 2(J + 1)^3(2J + 1)
\end{align*}

Table 3.8
\begin{table}[H]
    \centering
    \caption{Line strengths of triplet transitions for a Hund's case (b) $\state{3}{\Sigma}\dash\state{3}{\Sigma}$ transition.}
    \begin{tabular}{cc}
        \toprule
        Branches & Line Strength \\
        \midrule
        $P_1(J)$ & $\dfrac{J[(J^2 - 1)u'^+_1(J - 1)u''^+_1(J) + (J^2 - 1)u'^-_1(J - 1)u''^-_1(J) + 8J^2(J - 1)^2]^2}{16C'_1(J - 1)C''_1(J)}$ \\
        \addlinespace[0.5em]
        $Q_1(J)$ & $\dfrac{(2J + 1)[-J(J + 1)u'^+_1(J)u''^+_1(J) + J(J + 1)u'^-_1(J)u''^-_1(J)]^2}{16J(J + 1)C'_1(J)C''_1(J)}$ \\
        \addlinespace[0.5em]
        $R_1(J)$ & $\dfrac{(J + 1)^2[J(J + 2)u'^+_1(J + 1)u''^+_1(J) + J(J + 2)u'^-_1(J + 1)u''^-_1(J) + 8J^2(J + 1)^2]^2}{16(J + 1)C'_1(J + 1)C''_1(J)}$ \\
        \bottomrule
    \end{tabular}
\end{table}

\chapter{Temporary}

\section{Term Values}

\section{H\"onl-London Factors}

\section{Spectral Lineshapes}

\subsection{Lorentzian Profile}

From \textit{A Student's Guide to Atomic Physics} by Fox, eq. (3.31) and \textit{Spectroscopy and Optical Diagnostics for Gases} by Hanson, eq. (8.7):
\begin{equation*}
    L(\nu) = \frac{1}{2\pi}\frac{\adif{\nu}}{(\nu - \nu_0)^2 + (\adif{\nu}/2)^2}
\end{equation*}

From \textit{A Student's Guide to Atomic Physics} by Fox, eq. (3.32) and \textit{Spectroscopy and Optical Diagnostics for Gases} by Hanson, eq. (8.6):
\begin{equation*}
    \adif{\nu} = \frac{1}{2\pi}\ab(\frac{1}{\tau'} + \frac{1}{\tau''})
\end{equation*}

\subsection{Natural Broadening}

From \textit{Spectroscopy and Optical Diagnostics for Gases} by Hanson, eq. (8.11):
\begin{equation*}
    \adif{\nu}_N = \frac{1}{2\pi}\ab(\sum_k A_{ik} + \sum_k A_{jk})
\end{equation*}

\subsection{Collisional Broadening}

From \textit{A Student's Guide to Atomic Physics} by Fox, eq. (3.35) and \textit{Spectroscopy and Optical Diagnostics for Gases} by Hanson, eq. (8.17):
\begin{equation*}
    \adif{\nu}_C = \frac{p\sigma_c}{2\pi}\sqrt{\frac{8}{\pi\mu_{12}k_BT}}
\end{equation*}

From \textit{Spectroscopy and Optical Diagnostics for Gases} by Hanson, eq. (8.12) and Plasma Physics Slideset 5, page 5:
\begin{equation*}
    \sigma_c = \pi d_{12}^2 = \pi\frac{d_1 + d_2}{2}
\end{equation*}

From \textit{Spectroscopy and Optical Diagnostics for Gases} by Hanson, eq. (8.15):
\begin{equation*}
    \mu_{12} = \frac{m_1m_2}{m_1 + m_2}
\end{equation*}

\subsection{Gaussian Profile}

From \textit{A Student's Guide to Atomic Physics} by Fox, eq. (3.42) and \textit{Spectroscopy and Optical Diagnostics for Gases} by Hanson, eq. (8.22):
\begin{equation*}
    G(\nu) = \frac{2}{\adif{\nu}}\sqrt{\frac{\ln{2}}{\pi}}\exp\ab[-4\ln{2}\ab(\frac{\nu - \nu_0}{\adif{\nu}})^2]
\end{equation*}

\subsection{Doppler Broadening}

From \textit{A Student's Guide to Atomic Physics} by Fox, eq. (3.43) and \textit{Spectroscopy and Optical Diagnostics for Gases} by Hanson, eq. (8.24):
\begin{equation*}
    \adif{\nu}_D = \nu_0\sqrt{\frac{8kT\ln{2}}{mc^2}}
\end{equation*}

\section{Partition Functions}

\section{Rate Equations}

\appendix
\chapter{Diatomic Constants}
\label{a:diatomic_constants}

\begin{table}[H]
    \centering
    \caption{Diatomic constants for $^{16}\mathrm{O}_2$ \cite{nist:diatomic}.}
    \label{t:diatomic_constants_for_o2}
    \begin{tabular}{cccc}
        \toprule
        Symbol        & \multicolumn{2}{c}{State} & Units                                      \\
        \cmidrule(lr){2-3}
                      & $X~^3\Sigma_g^-$          & $B~^3\Sigma_u^-$        &                  \\
        \midrule
        \multicolumn{4}{c}{\textit{Electronic}}                                                \\
        \cmidrule(lr){1-4}
        $T_e$         & $0$                       & $49793.28$              & $\unit{cm^{-1}}$ \\
        \multicolumn{4}{c}{\textit{Vibrational}}                                               \\
        \cmidrule(lr){1-4}
        $\omega_e$    & $1580.19_3$               & $709.31$                & $\unit{cm^{-1}}$ \\
        $\omega_ex_e$ & $11.98_1$                 & $10.65$                 & $\unit{cm^{-1}}$ \\
        $\omega_ey_e$ & $0.0474_7$                & $-0.139$                & $\unit{cm^{-1}}$ \\
        $\omega_ez_e$ & $-0.00127_3$              & $-$                     & $\unit{cm^{-1}}$ \\
        \multicolumn{4}{c}{\textit{Rotational}}                                                \\
        \cmidrule(lr){1-4}
        $B_e$         & $[1.4376766]$             & $0.8190_2$              & $\unit{cm^{-1}}$ \\
        $\alpha_e$    & $0.0159_3$                & $0.01206$               & $\unit{cm^{-1}}$ \\
        $\gamma_e$    & $-$                       & $-5.5_6\times\num{e-4}$ & $\unit{cm^{-1}}$ \\
        $\delta_e$    & $-$                       & $-$                     & $\unit{cm^{-1}}$ \\
        \multicolumn{4}{c}{\textit{Centrifugal Distortion}}                                    \\
        \cmidrule(lr){1-4}
        $D_e$         &                           &                         & $\unit{cm^{-1}}$ \\
        $\beta_e$     &                           &                         & $\unit{cm^{-1}}$ \\
        \multicolumn{4}{c}{\textit{Spin-Splitting}}                                            \\
        \cmidrule(lr){1-4}
        $\lambda$     &                           &                         & $\unit{cm^{-1}}$ \\
        $\gamma$      &                           &                         & $\unit{cm^{-1}}$ \\
        \multicolumn{4}{c}{\textit{Other}}                                                     \\
        \cmidrule(lr){1-4}
        $H_e$         &                           &                         & $\unit{cm^{-1}}$ \\
        $r_e$         &                           &                         & \AA              \\
        $\nu_{00}$    &                           &                         & $\unit{cm^{-1}}$ \\
        \bottomrule
    \end{tabular}
\end{table}

\chapter{Notation for Diatomic Constants}
\label{a:notation_for_diatomic_constants}

\begin{table}[H]
    \centering
    \caption{Notation for diatomic constants \cite{herzberg:diatomic,nist:sigma1,nist:sigma3}.}
    \label{t:notation}
    \begin{tabular}{clc}
        \toprule
        Symbol        & Definition                                                                  & Units            \\
        \midrule
        \multicolumn{3}{c}{\textit{Electronic}}                                                                        \\
        \cmidrule(lr){1-3}
        $T_e$         & Minimum electronic energy                                                   & $\unit{cm^{-1}}$ \\
        \multicolumn{3}{c}{\textit{Vibrational}}                                                                       \\
        \cmidrule(lr){1-3}
        $G$           & Vibrational energy                                                          & $\unit{cm^{-1}}$ \\
        $\omega_e$    & Vibrational constant - first term                                           & $\unit{cm^{-1}}$ \\
        $\omega_ex_e$ & Vibrational constant - second term                                          & $\unit{cm^{-1}}$ \\
        $\omega_ey_e$ & Vibrational constant - third term                                           & $\unit{cm^{-1}}$ \\
        $\omega_ez_e$ & Vibrational constant - fourth term                                          & $\unit{cm^{-1}}$ \\
        \multicolumn{3}{c}{\textit{Rotational}}                                                                        \\
        \cmidrule(lr){1-3}
        $B_e$         & Rotational constant - equilibrium                                           & $\unit{cm^{-1}}$ \\
        $\alpha_e$    & Rotational constant - first term                                            & $\unit{cm^{-1}}$ \\
        $\gamma_e$    & Rotational constant - second term (rotation-vibration interaction constant) & $\unit{cm^{-1}}$ \\
        $\delta_e$    & Rotational constant - third term                                            & $\unit{cm^{-1}}$ \\
        \multicolumn{3}{c}{\textit{Centrifugal Distortion}}                                                            \\
        \cmidrule(lr){1-3}
        $D_e$         & Centrifugal distortion constant - equilibrium                               & $\unit{cm^{-1}}$ \\
        $\beta_e$     & Centrifugal distortion constant - first term                                & $\unit{cm^{-1}}$ \\
        \multicolumn{3}{c}{\textit{Spin-Splitting}}                                                                    \\
        \cmidrule(lr){1-3}
        $\lambda$     & Spin-spin coupling parameter                                                & $\unit{cm^{-1}}$ \\
        $\gamma$      & Spin-rotation coupling parameter                                            & $\unit{cm^{-1}}$ \\
        \multicolumn{3}{c}{\textit{Other}}                                                                             \\
        \cmidrule(lr){1-3}
        $H_e$         & Sixth-order rotational constant                                             & $\unit{cm^{-1}}$ \\
        $r_e$         & Equilibrium internuclear distance                                           & \AA              \\
        $\nu_{00}$    & Position of $0\dash0$ band                                                  & $\unit{cm^{-1}}$ \\
        \bottomrule
    \end{tabular}
\end{table}

\chapter{Quantum Numbers}
\label{a:quantum_numbers}

\begin{table}[H]
    \centering
    \caption{Various quantum numbers.}
    \label{t:quantum_numbers}
    \begin{tabular}{clc}
        \toprule
        Symbol    & Definition                            & Values                                      \\
        \midrule
        \multicolumn{3}{c}{\textit{Single Electron in Atoms}}                                           \\
        \cmidrule(lr){1-3}
        $n$       & Principal                             & $1, 2, \dotsb$                              \\
        $l$       & Azimuthal                             & $0, 1, \dotsb, (n - 1)$                     \\
        $m_l$     & Magnetic                              & $-l, \dotsb, l$                             \\
        $m_s$     & Spin                                  & $\pm \frac{1}{2}$                           \\
        \multicolumn{3}{c}{\textit{Single Electron in Molecules}}                                       \\
        \cmidrule(lr){1-3}
        $\lambda$ & Orbital Angular Momentum              & $\abs{m_l}$                                 \\
        \multicolumn{3}{c}{\textit{Whole Atoms}}                                                        \\
        \cmidrule(lr){1-3}
        $S$       & Resultant Spin                        & $\sum s_i$                                  \\
        $L$       & Resultant Orbital Angular Momentum    & $\sum l_i$                                  \\
        $J$       & Total Angular Momentum                & $(L + S), (L + S - 1), \dotsb, \abs{L - S}$ \\
        $I$       & Nuclear Spin                          & ?                                           \\
        $F$       & Total Angular Momentum w/ Spin        & $(J + I), (J + I - 1), \dotsb, \abs{J - I}$ \\
        \multicolumn{3}{c}{\textit{Whole Molecules}}                                                    \\
        \cmidrule(lr){1-3}
        $M_L$     & ?                                     & $L, L - 1, \dotsb, -L$                      \\
        $\Lambda$ & Electronic Orbital Angular Momentum   & $0, 1, \dotsb, L$                           \\
        $\Sigma$  & ?                                     & $S, S - 1, \dotsb, -S$                      \\
        $\Omega$  & Resultant Electronic Angular Momentum & $\abs{\Lambda + \Sigma}$                    \\
        $N$       & Total Angular Momentum w/o Spin       & $\Lambda, \Lambda + 1, \dotsb$              \\
        \bottomrule
    \end{tabular}
\end{table}

\chapter{States}
\label{a:states}

\begin{table}[H]
    \centering
    \caption{Various atomic and molecular states.}
    \label{t:states}
    \begin{tabular}{cc}
        \toprule
        Defining Quantum Number & Values                                 \\
        \midrule
        \multicolumn{2}{c}{\textit{Single Electron in Atoms}}            \\
        \cmidrule(lr){1-2}
        $l$                     & s, p, d, f, $\dotsb$                   \\
        \multicolumn{2}{c}{\textit{Single Electron in Molecules}}        \\
        \cmidrule(lr){1-2}
        $\lambda$               & $\sigma, \pi, \delta, \varphi, \dotsb$ \\
        \multicolumn{2}{c}{\textit{Whole Atoms}}                         \\
        \cmidrule(lr){1-2}
        $L$                     & S, P, D, F, $\dotsb$                   \\
        \multicolumn{2}{c}{\textit{Whole Molecules}}                     \\
        \cmidrule(lr){1-2}
        $\Lambda$               & $\Sigma, \Pi, \Delta, \Phi, \dotsb$    \\
        \bottomrule
    \end{tabular}
\end{table}

\printbibliography
\addcontentsline{toc}{chapter}{Bibliography}

\end{document}
